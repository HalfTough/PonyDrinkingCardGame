\documentclass[notitlepage]{article}
\usepackage{tabu}
\usepackage{hyperref}
\pagestyle{empty}
\title{Pony Drinking Card Game}

\begin{document}
\maketitle
\section{Introduction}
I will start by saying something very important: \textsl{Don't be an idiot}. Alcohol can be dangerous. In fact, for many of you, this might be the most dangerous game you'll ever play. It's good if it is. So, don't drink if you're pregnant, minor, driver or shouldn't do it for any other reason. And when you do drink, do it responsibly to have fun, not to poison yourself with more liquid your body can handle. If we cleared that up, let's hop to something more plesant.

This game is free. I mean really, totally free. Not only it costs nothing but materials to play, but you can use every bit of it's content in any way you want (that is as long it's intelectual property of people who made this game not people who own \textsl{My Little Pony}). You can copy it, share it, edit it, build your own game or whatever you want upon it. You can also contribute, but it's up to me whether I accept your ideas or not. But, hey, don't let that stop you from creating your own vision. You can find tool for generating your own cards or editing existing ones on \url{https://github.com/HalfTough/PonyDrinkingCardGame}.

\section{Preperations}
It's a card game, so you're gonna need cards. Very special cards in fact. If you don't have them, go download whole deck at URL GOES HERE.

Next thing you need is alcohol. Pretty much any type will be fine. Except for cider. In this game, cider is special drink and you're gonna need it aside from your basic alcohol. On top of that you're gonna need some \textsl{sweet alcohol}. It can be something you prepare specially for this purpose like fruit beer, mead or some nice drink. Or just your basic alcohol with something sweet added to it.

There are also some items that might be used by certain cards. Items in question are: stopwatch, marker pen (or regular pen), paper, blindfold, coin, dice or some other random number generator, alcohol-free drink and of course source of music. Some of those are less important than others. As you play, you'll figure out what you need and what can be replaced one way or another.

Finally, and I know this one can be difficult, you need friends. Not only that, you need adult friends willing to play drinking game themed around \textsl{My Little Pony}. It'd be nice to have at least three people. So, yeah. Go use those friendship skills you learned from the show.

\section{Basic rules}
TL;DR: Draw a card, do what is says, next player draws a card... But do read \textsl{Before the game starts} section.

Shuffle all cards and put them face down in one place, preferably at the center of playfield, so everyone can access it easily. We shall call those cards THE DECK. You should also have place called DISCARD PILE. It's where all used cards go.

Player who starts given turn will be called ACTIVE PLAYER and will remain active player until next turn starts, even if another player is making some decision at the time. When a turn starts, it's time for active player to draw a card from the deck. What happens next, depends on type of drawn card.

PONY CARDS are the blue ones with Pinkie Pie in the left upper corner. When player draws Pony Card, it becomes his or her Active Pony Card. Active Pony Card should be placed face up, near its player and it takes effect as long it stays in the game. Player can have only one Active Pony Card. When (s)he draws another Pony Card, it will become his/her new Active Pony Card and the old one goes on the Discard Pile.

ACTION CARDS are yellow and have Rainbow Dash in upper left corner. When you draw an Action Card, show it to everyone and follow instructions on the card. When effect ends, put it on the Discard Pile. If effect lasts for more than one turn, leave Action Card on the table as long as it's active.

Green cards with Fluttershy in left upper corner are FRIENDSHIP CARDS. They are the same way as Action Cards, but work on all players.

And finally, pink ones with Twilight are MAGIC CARDS. When you draw a Magic Card, place it in your hand without showing it to other players. You can use it at any time, as long it makes sense.

Once Active Player finishes action, next turn starts and it's time for next player to draw a card. Round that started with player's turn ends right before this player's next turn. So, let's say we have players 1, 2, 3, 4, playing in that order. If player 3 draws a card having effect that lasts one round, it means that effecf will be active during player's 3, 4, 1 and 2 turns and will end when turn of player 2 ends. So if you have 5 players, a round will normally last 5 turns.

\section{Before the game starts}
There are some things, that should be decided before you start playing. You see, when the game says \textsl{"take a shot"}, it doesn't mean an actual shot. If it did, you'd die after 20 minutes of playing. Because this game can be played using various different kinds of alcohol and people have different preferences and durability, there's no one definition of what \textsl{a shot} is. You have to decide that before the game. For stronger liquids, each \textsl{shot} might gain you a token. Once you have set number of tokens, exchange them for an actual shot. At the end of this instruction, I attached a table that might help you with this.
%Have in mind, this game was tested on Poles. Poles who like \textsl{My Little Pony}, but Poles nevertheless. Table is just a suggestion. Don't feel obligated to stick to it.
(This is beta version of the game and I couldn't properly test it, so no table for now.)

Some cards mention \textsl{"a drink"}. Similarly to \textsl{a shot} you should define what it is before game starts. My tip is, one portion should have about 330ml and 5-20\% alcohol.

Card called \textsl{Filthy Rich} allows player to pay an actual money to cancel shots. Because money can have very different value depending on who you are, all players should decide on the cost before game starts. You might also increase cost in case of canceling multiple shots in one turn. For example price of each shot might be increased by 0.5, so canceling one might cost 1 money, but canceling 2 will cost 2.5 moneys (1+1.5) and canceling 3 would cost 4.5 (1+1.5+2).\newline
Aside from cost you should decide what to do with money after game is over. You might buy something for y'all or give it to charity. Donating it to the author of this game is also a very good idea. \\
Or if you don't like this idea, you can just remove \textsl{Filthy Rich} from your deck.

\section{Extra}
Most of the rules are pretty simple. But sometimes it might not be exactly clear what you should do at given time. Usually, simply doing what's seems most logical is enough. Also, interpreting something wrong won't exactly ruin the game. But if you do have doubts, this section should help you.

Sometimes card forces you to behave certain way but doesn't say what happens when you fail to do it. In that case answer is either: take a shot or take a shot per turn. Depending which one makes more sense.

Rarity\newline
What kind of trends can you make up? Be creative. You can make up something funny or challenging (like balancing shoe on your head), but it must be possible for all players to do (wearing t-shirt with dinosaur on it, when you're only person who has dinosaur t-shirt is not an option). Penalty for not following the trend is shot per turn.

Rainbow Dash\newline
Description reads: \textsl{When someone has to drink, take half of this burden on yourself. If shots are not divisible by 2, round it down.}. 
Let's say player A must drink 4 shots. Player with Rainbow Dash as Active Pony Card (I'll call him RD) now helps him. So A drinks 2 shots and RD drinks 2 shots. If A must drink 3 shot, RD takes 1 and A stays with 2. \\
\begin{tabular}{| c | c | c |}
\hline
Start & A & RD \\ \hline
1 & 1 & 0 \\ \hline
2 & 1 & 1 \\ \hline
3 & 2 & 1 \\ \hline
4 & 2 & 2 \\ \hline
5 & 3 & 2 \\ \hline
6 & 3 & 3 \\ \hline
\end{tabular} \\
Also, count each player separately. If there are four players, any everybody has one shot to drink, DR doesn't need to help (each 1/2 is rounded down to 0).

Starlight Glimmer\newline
\textsl{Each time someone drinks, everybody else drinks same amount.} What if multiple players must drink at the same time? Should everyone drink for each player or just drink once? The answer is once. Let's have some examples: \\
\begin{tabular}{| c | c |}
\hline
before & after \\ \hline
00010 & 11111 \\ \hline
01001 & 11111 \\ \hline
21011 & 22222 \\ \hline
\end{tabular}

Changeling\\
This works as if you had same card. If another player loses or changes Active Pony Card, your effect stays.

Stuff of Sameness\newline
Yes, I do know stuff was fake.

Hoof Wrestle\newline
If there are four of you, you have it easy. First round goes A~B, second C~D. If A and C win, third rund goes A~C. If A wins, A drinks nothing, C drinks one shot, B and D both drink two shots. Chances are however, there will be more (or less) players. In that case, you can take three players, have each face each and pick one winner. So you can have A~B~C and D~E in first two rounds. Make it work somehow. It's probably good idea to pick stating positions randomly.

Really Bald Donkey\newline
If you have two (or more) players who have (as far as you can tell) equally short hair, have them both drink.

Tardy\newline
This card doesn't apply to Magic Cards.\\
\begin{tabular}{| l | l |}
\hline
\textbf{Effect} & \textbf{After change} \\ \hline
Drink n when you do x & Drink 2n when you do x \\ \hline
Pay x & Pay 2x \\ \hline
If a drink n & If a drink 2n \\ \hline
Draw n cards & Draw 2n cards \\ \hline
Lasts n rounds & Lasts 2n rounds \\ \hline
Take/look up n cards from top & Take/look up 2n cards from top \\ \hline
Choose n cards from discard pile & Choose 2n cards from discard pile \\ \hline
Other & No change \\ \hline
\end{tabular}

First Rule\newline
Only exception to this card is \textsl{Hoof Wrestle}. What if you have to prepare a drink or something? Ask for help or do it \textsl{earth pony style}.

Rainbow Dash + Starlight Glimmer\newline
What if both Rainbow and Starlight are active at the same time? First equalize everybody and after that poor Rainbow helps everyone.

Applebloom/Scootaloo + Discord\newline
Treat name given by card as real name when it's being switched. So if we have players A B C D sitting in that order clockwise and B is Scootaloo and Discord is in play, C is now called Scootaloo. If A has Applebloom and A calls D big sister, A should call himself/herself big sister.

Changeling + Rainbow Dash\\
Instead of dividing shots between two people, you divide them between three.
\begin{tabular}{| c | c | c | c |}
\hline
Start & A & RD1 & RD2 \\ \hline
1 & 1 & 0 & 0 \\ \hline
2 & 1 & 1 & 0 \\ \hline
3 & 1 & 1 & 1 \\ \hline
4 & 2 & 1 & 1 \\ \hline
5 & 2 & 2 & 1 \\ \hline
6 & 2 & 2 & 2 \\ \hline
\end{tabular}\\
RD1 is a player who sits closest to player A counting clockwise. It doesn't matter if it's original Rainbow Dash or a Changeling.

Got more questions? Contact me. I will help and possibly add more tips to this section.
\end{document}
